

\documentclass[a4paper,12pt]{article} % добавить leqno в [] для нумерации слева


\usepackage[utf8]{inputenc}			% кодировка исходного текста
\usepackage[english,russian]{babel}	% локализация и переносы

%%% Дополнительная работа с математикой
\usepackage{amsmath,amsfonts,amssymb,amsthm,mathtools} % AMS
\usepackage{icomma} % "Умная" запятая: $0,2$ --- число, $0, 2$ --- перечисление

%% Номера формул
\mathtoolsset{showonlyrefs=true} % Показывать номера только у тех формул, на которые есть \eqref{} в тексте.

%% Шрифты
\usepackage{euscript}	 % Шрифт Евклид
\usepackage{mathrsfs} % Красивый матшрифт

\DeclareMathOperator{\sgn}{\mathop{sgn}}
\DeclareMathOperator{\sgn}{sign}
\DeclareMathOperator{\sign}{sign}
\DeclareMathOperator{\grad}{grad}
\DeclareMathOperator{\card}{card}
\DeclareMathOperator{\Lin}{\mathrm{Lin}}
\DeclareMathOperator{\Linp}{\Lin^{\perp}}
\DeclareMathOperator*\plim{plim}

%% Перенос знаков в формулах (по Львовскому)
\newcommand*{\hm}[1]{#1\nobreak\discretionary{}
{\hbox{$\mathsurround=0pt #1$}}{}}


\DeclareMathOperator*{\argmin}{arg\,min}
\DeclareMathOperator*{\argmax}{arg\,max}
\DeclareMathOperator*{\amn}{arg\,min}
\DeclareMathOperator*{\amx}{arg\,max}
\DeclareMathOperator{\cov}{Cov}
\DeclareMathOperator{\Var}{Var}
\DeclareMathOperator{\Cov}{Cov}
\DeclareMathOperator{\Corr}{Corr}
\DeclareMathOperator{\pCorr}{pCorr}
\DeclareMathOperator{\E}{\mathbb{E}}
\let\P\relax
\DeclareMathOperator{\P}{\mathbb{P}}

\newcommand{\cN}{\mathcal{N}}
\newcommand{\cU}{\mathcal{U}}
\newcommand{\cBinom}{\mathcal{Binom}}
\newcommand{\cBin}{\cBinom}
\newcommand{\cPois}{\mathcal{Pois}}
\newcommand{\cBeta}{\mathcal{Beta}}
\newcommand{\cGamma}{\mathcal{Gamma}}
\newcommand \N{\mathbb{N}}
\newcommand \R{\mathbb{R}}
\newcommand \Z{\mathbb{Z}}
\newcommand{\dx}[1]{\,\mathrm{d}#1} 
\newcommand{\ind}[1]{\mathbbm{1}_{\{#1\}}} 
\newcommand{\eqdef}{\mathrel{\stackrel{\rm def}=}}
\newcommand{\iid}{\mathrel{\stackrel{\rm i.\,i.\,d.}\sim}}
\newcommand{\const}{\mathrm{const}}
\renewcommand{\le}{\leqslant}
\renewcommand{\ge}{\geqslant}
\renewcommand{\leq}{\leqslant}
\renewcommand{\geq}{\geqslant}

%%% Заголовок
\author{Айдана Муратова}
\title{Домашнее задание}

\begin{document}

\maketitle

\textbf{Промежуточный экзамен 2017-2018}

\begin{enumerate}

    \item

    \[ \Var(X) = \E(X^2) - (\E(X))^2 \]
    \[ (\E(X^2) = \Var(X) + \E(X))^2 = 10\]
    
    $\P(X^2 \geq 100)$ 
    $X^2 \geq 0$ всегда, используем эквивалентную формулу из следствия неравенства Маркова.
    
    \[\P(X^2 \geq 100) \leq \frac{\E(X^2)}{100} = 0.1 \] - верхняя граница диапозона
    
    Следовательно, $\P(X^2 \geq 100)$ принадлежит $[0, 0.1]$

    Ответ: A

    \item

    \[ \E(\xi) =\lambda \]
    \[ \Var\xi) =\lambda \]
    
    \[ \Var(\xi) = \E(\xi^2) - (\E(\xi))^2 \]
    \[ \E(\xi^2) = \Var(\xi) + (\E(\xi))^2 = \lambda + \lambda^2 = \lambda \cdot (1 + \lambda)\]

    Ответ: E

    \item
    
    \[ \Corr(X+Y,Y) = \frac{\Cov(X+Y,Y)}{\sqrt{\Var(X+Y)\cdot \Var(Y)}} = \frac{6}{\sqrt{7\cdot 9}} = \frac{2}{\sqrt{7}} \]

    \[ \Var(X+Y) = \Var(X) + \Var(Y) + 2 \cdot \Cov(X,Y) = 4 + 9 + 2 \cdot (-3) = 7 \]

    \[\Cov(X+Y,Y) = \Cov(X,Y) + \Cov(Y,Y) = -3 + 9 = 6\]

    Ответ: C
    
    \item
    
    Функция плотности для случайной величины с нормальным распределением:
    
    \[ f(x)=\frac{1}{\sqrt{2\pi\sigma^2}} e^{-\frac{(x-\mu)^2}{2\sigma^2}} \]
    
    $\sigma=1$ 
    $\mu=0$

    Ответ: B

    \item

    \[f_{X,Y}(1,1) = \frac{1}{S} = \frac{1}{\frac{1}{2}\cdot 4\cdot 2} = \frac{1}{4}\]

    Ответ: A

    \item
    
    
    События A, B и C независимы в совокупности, если
    
    \[ \P(ABC) = \P(A)\P(B)\P(C) \]

    Ответ: B

    \item

    Построили график функции плотности $\xi$ 
    прямоугольник с h = $\frac{1}{4}$
    Интеграл от $-\infty$ до $+\infty$ от функции плотности равен единице, следовательно, площадь всего прямоугольника равна единице:
    
    \[\P({\xi\in[3,6]}) = \frac{1}{4}\]

    Ответ: B
    
    \item

    $X, Y$ – случайные величины

    \[\P(X=-5) = \ldots = \P(X=5) = \frac{1}{11}\]

    \[\P(Y=-1) = \P(Y=0) = \P(Y=1) = \frac{1}{3}\]

    Так как $X + Y^2 = 2$, то допускаются следующие значения:
    \[Y = 1, X = 1\]
    \[Y = 0, X = 2\]
    \[Y = -1, X = 1\]

    Так как случайные величины независимые, следует: 
    \[\P(X + Y^2 = 2) = \frac{1}{11} \cdot\frac{1}{3} \cdot 3 = \frac{1}{11}\]

    Ответ: C

    \item
    
    Кол-во секторов:

    \[\frac{2\pi}{\frac{\pi}{3}} = 6\]
    
    Зачения точек круга равновероятны, следовательно:

    \[\P(\text{<<красный>>}) = \frac{1}{6}\]

    Ответ: E

    \item
    
    \[\P(A\cup B) = \P(A) + \P(B) - \P(A\cap B)\]

    \[0.6 = 0.3 + \P(B) - 0.2\]
    
    Следует:

    \[\P(B) = 0.5\]

    Ответ: B

    \item
    
    Если $a$, $b$, $c$ —  константы, $X$, $Y$ — случайные величины:
    \[\Var(aX + bY + c) = a^2 \Var(X) + b^2 \Var(Y) + 2ab\Cov(X,Y)\]
    
    Тогда:

    \[\Var(2X - Y + 1) = 4\cdot \Var(X) + \Var(Y) - 4 \cdot \Cov(X,Y) \]
    
    \[\Var(2X - Y + 1) = 4 \cdot 4 + 9 - 4 \cdot (-3) = 37\]

    Ответ: B

    \item
    
    По закону больших чисел:

    \[\plim _{n\rightarrow +\infty}\frac{X_{1}^2 + \dots + X_{n}^2}{n} = \E(X^2) = \Var(X) +(\E(X))^2 = 1\]

    Ответ: B

    \item

    Условная функция плотности:

    \[f\left(x\mid y=\frac{1}{2}\right) = \frac{f\left(x,\frac{1}{2}\right)}{f_{y}\left(\frac{1}{2}\right)} = \frac{6x\cdot\frac{1}{4}}{\frac{3}{4}} = 2x\]

    \[f_{y}(y) = \int_0^1 6\cdot x\cdot y^2 dx = \left.3 \cdot x^2 \cdot y^2\right|_0^1  = 3\cdot y^2, y \in [0;1] \]
    
    \[f_{y}\left(\frac{1}{2}\right) = 3 \cdot \left(\frac{1}{2}\right)^2 = \frac{3}{4}\]

    Ответ: C

    \item

    Неизвестно, чему равно $n$. Можно решить методом подбора
    
    Ответ: ?

    \item

    \[\Cov(X+2Y, 2X + 3) = \Cov(X+2Y, 2X) = \Cov(X, 2X) + \Cov(2Y, 2X) \]
    \[\Cov(X+2Y, 2X + 3) = 2 \cdot \Cov(X,X) + 4 \cdot \Cov(X,Y) = 2 \cdot 4 + 4 \cdot (-3) = -4\]

    Ответ: A

    \item

    \[\E((X-1)Y) = \E(XY - Y) = \E(XY) - \E(Y) = \Cov(X,Y) + \E(X)\cdot\E(Y) -\E(Y)\] 
    \[\E((X-1)Y) = -3 + (-2) - 2 = -7\]

    Ответ: B

    \item
    
    $\P(X_{i} = 1) = \frac{1}{6}$

    $\P(X_{i} = 0) = \frac{5}{6}$

    \[\P(X_{1} + X_{2} = 1) = \P(X_{1} = 0, X_{2} = 1) + \P(X_{1} = 1, X_{2} = 0) \]
    \[\P(X_{1} + X_{2} = 1) = \frac{5}{6} \cdot \frac{1}{6} + \frac{1}{6} \cdot \frac{5}{6} = \frac{10}{36}\]
    
    \[\P(X_{1} = 0\mid X_{1} + X_{2} = 1) = \frac{\P(X_{1} = 0\cap X_{1} + X_{2} = 1)}{X_{1} + X_{2} = 1}) = \frac{1}{2}\]
    
    Ответ: B

    \item

    \[X + Y \sim \N(\E(X) + \E(Y), \Var(X) + \Var(Y))\]

    \[X + Y \sim \N(3,7)\]

    \[\P(X + Y\leq 3) = \P\left(\frac{X+Y-3}{\sqrt{7}} < \frac{3-3}{\sqrt{7}}\right) = (\Z\leq 0) = \frac{1}{2}\]

    Ответ: C

    \item

    \[\P(i=1,2,3\mid \text{<<6>>}) = \frac{\P(i=1,2,3\cap \text{<<6>>})}{\P(\text{<<6>>})} = \frac{\frac{3}{5}\cdot \frac{1}{6}}{\frac{1}{5}\cdot \frac{1}{6} + \frac{1}{5}\cdot \frac{1}{6} + \frac{1}{5}\cdot \frac{1}{6} + \frac{1}{5}\cdot \frac{1}{2} + \frac{1}{5}\cdot \frac{1}{10}} = \frac{\frac{3}{30}}{\frac{11}{50}} = \frac{5}{11} \]

    Ответ: C

    \item

    A) $1\cdot1 - 2\cdot2 < 0$

    B) $1\cdot9 - 4\cdot4 < 0$

    C) $9\cdot6 - 7\cdot7 > 0$
    
    D) отрицательная
    
    E) несимметричная

    Ответ: C

    \item

    \[\E(\alpha X + (1 - \alpha)Y) = \alpha \E(X) + (1-\alpha) \E(Y) = -\alpha + 2\cdot(1-\alpha) = 0\]
    \[2 - 3\cdot\alpha = 0\]
    \[\alpha = \frac{2}{3}\]

    Ответ: A

    \item

    \[\P(\xi = 0) = (1-p)^n = \frac{1}{4^2} = \frac{1}{16}\]

    Ответ: B

    \item
 
    \[\P(X\geq1) = 1 - \P(k=0) = 1 - e^{-4}\]

    Ответ: C

    \item
    
    \[\E(\xi^2) = \Var(\xi) + (\E(\xi))^2 = p\cdot (1-p) + p^2 = p\]

    Ответ: B

    \item

    \[\E(\xi) = \frac{1}{\lambda}\]
    \[\Var(\xi) = \frac{1}{\lambda^2}\]

    \[\E(\xi^2) = \Var(\xi) + (\E(\xi))^2 = \frac{1}{\lambda^2} + \frac{1}{\lambda^2} = \frac{2}{\lambda^2}\]

    Ответ: A

    \item

    Кол-во секторов:

    \[\frac{2\pi}{\frac{\pi}{3}} = 6\]

    \[\P(\text{<<красный>>}) = \P(\text{<<синий>>}) =\frac{1}{6}\]
    
    Ответ: E

    \item

    \[\E(XY) = \int_0^1\int_0^1 x\cdot y \cdot 6 \cdot x \cdot y^2 dxdy = \left.\int_0^1 2\cdot x^3 \cdot y^3 \right|_0^1 dy = \left.\frac{2\cdot y^4}{4}\right|_0^1 = \frac{1}{2}\]

    Ответ: A

    \item
    
    \[\Var(\alpha X + (1-\alpha) Y) = \alpha^2 \Var(X) + (1-\alpha)^2 \Var(Y) + 2\cdot \Cov(X,Y)\cdot\alpha\cdot(1-\alpha)\]
    \[\Var(\alpha X + (1-\alpha) Y) = 4\cdot\alpha^2 + 9\cdot(1-\alpha)^2 - 6\cdot\alpha\cdot(1-\alpha) = 4\cdot\alpha^2 + 9 - 18\cdot\alpha + 9\cdot\alpha^2 - 6\cdot\alpha + 6\cdot \alpha^2\]
    \[\Var(\alpha X + (1-\alpha) Y) = 19\cdot\alpha^2 - 24\cdot\alpha + 9\]
    
    Точка минимума:
    \[\alpha^{*} = \frac{24}{38} =\frac{12}{19}\]

    Ответ: нет верного ответа

    \item

    \[\P(\text{<<без багажа>>}) = \frac{1}{4}\]
    \[\P(\text{<<с рюкзаком>>}\mid\text{<<без багажа>>}) = 0.5\]
    \[\P(\text{<<с рюкзаком>>}\mid\text{<<c багажом>>}) = \frac{55}{150}\]

    \begin{multline*}
        \P(\text{<<без рюкзака>>}) = \P(\text{<<без рюкзака>>}\mid\text{<<без багажа>>})\P(\text{<<без багажа>>}) +\\+ \P(\text{<<без рюкзака>>}\mid\text{<<с багажом>>})\P(\text{<<с багажом>>}) = \frac{1}{2}\cdot\frac{1}{4} + \frac{95}{150} \cdot\frac{3}{4} = 0.6
    \end{multline*}
   

    Ответ: A

    \item

    \[\P(|X-2| \geq 10) \leq \frac{\Var(X)}{100}\]

    \[\P(|X-2| \leq 10) \geq 1 - \frac{\Var(X)}{100} = 0.94\]

    Ответ: C

\end{enumerate}

\end{document}